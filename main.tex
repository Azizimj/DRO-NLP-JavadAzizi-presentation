%Preamble
% \documentclass[handout]{beamer}
\documentclass{beamer}

% \usepackage{default}
% \usetheme{default}
% \usecolortheme{dove}

% section in the header
% \useoutertheme{miniframes}
% \useoutertheme{sidebar}

\usepackage{latexsym,xcolor,multicol,booktabs,calligra}


%Defining viterbi template colors
\definecolor{viterbiyellow}{HTML}{E9A000}
\definecolor{viterbired}{HTML}{990000}

%Redefine more fields as required
\newcommand{\viterbititle}[1]{\title{\textcolor{viterbiyellow}{#1}}}
\newcommand{\viterbiauthor}[1]{\author{\textcolor{viterbiyellow}{#1}}}
\newcommand{\viterbidate}[1]{\date{\textcolor{viterbiyellow}{#1}}}

%Change this environment to include any additional formatting required
\newenvironment{viterbiframe}[1]{
\begin{frame}{\textcolor{viterbired}{#1}}
}
{\end{frame}}

%To show the outline slide before each section
\AtBeginSection[]
{
  \begin{viterbiframe}{Outline}
    \tableofcontents[currentsection,currentsubsection]
  \end{viterbiframe}
}


\setbeamertemplate{headline}
{%
\begin{beamercolorbox}{section in head/foot}
\vskip2pt\insertnavigation{\paperwidth}\vskip2pt
\end{beamercolorbox}%
}

\setbeamertemplate{footline}
{
  \leavevmode%
  \hbox{%
%   \begin{beamercolorbox}[wd=.4\paperwidth,ht=2.25ex,dp=1ex,center]{author in head/foot}%
%     \usebeamerfont{author in head/foot}\insertshortauthor
%   \end{beamercolorbox}%
  \begin{beamercolorbox}[wd=.6\paperwidth,ht=2.25ex,dp=1ex,center]{title in head/foot}%
    \usebeamerfont{title in head/foot}\insertshorttitle\hspace*{3em}
    \insertframenumber{} / \inserttotalframenumber\hspace*{1ex}
  \end{beamercolorbox}}%
  \vskip0pt%
}


% Package to manage the bibliography
% \usepackage[backend=biber, style=numeric, sorting=none]{biblatex}
\usepackage[style=verbose,backend=biber]{biblatex}
% \addbibresource{refsSagawa20.bib}
% \bibliographystyle{plain}
\bibliography{refsSagawa20,refs}


\setbeamerfont{footnote}{size=\tiny}


% \usepackage{enumitem}

\usepackage{cleveref}
\usepackage{hyperref}

% %end preamble


% \usepackage{floatrow}
\usepackage{hyperref}
% \usepackage{url}
\usepackage{booktabs}       % professional-quality tables
\usepackage{amsfonts}       % blackboard math symbols
\usepackage{nicefrac}       % compact symbols for 1/2, etc.
\usepackage{xcolor}
\usepackage{amsmath}
\usepackage{amssymb}
\usepackage{amsthm}
\usepackage{caption}
\usepackage{subcaption}
\usepackage{mathtools}
\usepackage[algoruled]{algorithm2e}
\usepackage{multirow}
% \usepackage{tabularx}

% Paper-specific macros

\DeclareMathOperator*{\argmax}{arg\,max}
\DeclareMathOperator*{\argmin}{arg\,min}

\newcommand{\Ltwo}{\ell_2}
\newcommand{\hP}{{\hat{P}}}
\newcommand{\hterm}{{\hat\theta_{\text{ERM}}}}
\newcommand{\htdro}{{\hat\theta_{\text{DRO}}}}
\newcommand{\htadj}{{\hat\theta_{\text{adj}}}}
\newcommand{\htq}{{\hat\theta_{q}}}
\newcommand{\htQ}{{\hat\theta_{Q}}}
\newcommand{\htw}{{\hat\theta_{w}}}

\newcommand\gset{\sG}
\newcommand\Pg{P_g}
\newcommand\padv{q}
\newcommand\ngroup{m}
\newcommand\iter[1]{{^{(#1)}}}
\newcommand\condprisk[1]{\E_{X,Y\sim\Pg}\left[\ell(#1; (X,Y)) \right]}
\newcommand\gengap{\delta_{g}}
\newcommand\hatgengap{\hat{\delta}_{g}}
\newcommand{\xp}{x^\prime}
\newcommand{\yp}{y^\prime}
\newcommand{\gp}{g^\prime}

\newcommand{\Btheta}{B_\Theta}
\newcommand{\Bgrad}{B_\nabla}
\newcommand{\Bloss}{B_\ell}

\newcommand\todo[1]{}
\newcommand\pw[1]{}
\newcommand\shiori[1]{}
\newcommand\pl[1]{}
\newcommand\thc[1]{}
%
% \newcommand\todo[1]{\textcolor{red}{[TODO: #1]}}
% \newcommand\pw[1]{\textcolor{blue}{[PW: #1]}}
% \newcommand\pl[1]{\textcolor{red}{[PL: #1]}}
% \newcommand\thc[1]{\textcolor{brown}{[TH: #1]}}
% \newcommand\shiori[1]{\textcolor{teal}{[SS: #1]}}

% \newfloatcommand{capbtabbox}{table}[][\FBwidth]

\makeatletter
\newtheorem*{rep@theorem}{\rep@title}
\newcommand{\newreptheorem}[2]{%
\newenvironment{rep#1}[1]{%
 \def\rep@title{#2 \ref{##1}}%
 \begin{rep@theorem}}%
 {\end{rep@theorem}}}
\makeatother



% \newtheorem{lemma}{Lemma}
\newtheorem{proposition}{Proposition}
% \newtheorem{theorem}{Theorem}
\newtheorem{defn}{Definition}
\newtheorem{counterexample}{Counterexample}
\newtheorem{conjecture}{Conjecture}
% \newtheorem{corollary}{Corollary}
\newtheorem{assumption}{Assumption}
\newreptheorem{lemma}{Lemma}
\newreptheorem{proposition}{Proposition}
\newreptheorem{theorem}{Theorem}
\newreptheorem{defn}{Definition}
\newreptheorem{conjecture}{Conjecture}
\newreptheorem{corollary}{Corollary}
\newreptheorem{assumption}{Assumption}

\newenvironment{proofsketch}{%
  \renewcommand{\proofname}{Proof sketch}\proof}{\endproof}
\newtheorem*{remark}{Remark}

\input{math-macro}
% \input{std-macros}

%Document Begins here
\begin{document}


%Title slide

\viterbititle{Fairness IV: Distributionally robust optimization}

\viterbiauthor{\small Mohammad Javad Azizi}
\viterbidate{\small Spring 2022}

\setbeamertemplate{background}{\includegraphics[height=\paperheight]{img/title.png}}


\begin{frame}
\titlepage
\end{frame}

\setbeamertemplate{background}{\includegraphics[height=\paperheight]{img/bg3.png}}


%Table of contents
%Each \section or \subsection command in the document generates a TOC entry
\begin{viterbiframe}{Outline}
\tableofcontents
\end{viterbiframe}

%the document can be split in terms of sections or slides. I would prefer section level as a good tradeoff between number of tex files and complexity of each slide.
% \input{slide1.tex}


\section{DR NNs}
\begin{viterbiframe}{{Paper Intro}}
\begin{itemize}
    \item {\bf Distributionally Robust Neural Networks for Group Shifts: On the Importance of Regularization for Worst-Case Generalization}
    \begin{itemize}
        {\small
        \item[] Shiori Sagawa, \textit{\scriptsize Stanford University}
        \item[] Pang Wei Koh, \textit{\scriptsize Stanford University}
        \item[] Tatsunori B. Hashimoto, \textit{\scriptsize Microsoft}
        \item[] Percy Liang, \textit{\scriptsize Stanford University}}
    \end{itemize}
\end{itemize}
% We follow the current of the paper.
\end{viterbiframe}

% \section{Introduction (DR NNs)}
\begin{viterbiframe}{Motivation}

\begin{itemize}
    \begin{columns}
    \begin{column}{.6\textwidth}
    \item Typical goal: High accuracy on an i.i.d. test set
    \pause
    \item Approach: Minimize the \alert{average loss} on a training set 
    \pause
    \item Caveat: Fail on rare and atypical examples 
    % \footcite{demszky2018transforming}
    \begin{itemize}
        \item Rely on \alert{spurious} correlations: misleading heuristics that work for most training examples but \alert{do not} always hold.
        \item Violate \alert{equity} considerations 
        % \footcite{buolamwini2018gender}
    \end{itemize}
    \end{column}
    \pause
    \begin{column}{.3\textwidth}
    \begin{figure}
        \centering
        \includegraphics[width=\linewidth]{img/seagel.jpg}
        % \caption{Caption}
        % \label{fig:my_label}
    \end{figure}
    \end{column}
    \end{columns}
    
\end{itemize}

\end{viterbiframe}

\begin{viterbiframe}{{\alert{Spurious} correlations}}
    
    \begin{itemize}
        \item Determining if two sentences agree or contradict
        \item The presence of negation words like ‘\alert{never}’ is strongly correlated with contradiction 
        \item Artifacts in crowdsourced training data
    \end{itemize}
    
\end{viterbiframe}

\begin{viterbiframe}{{\alert{Spurious} correlations}}
    \begin{figure}
        \centering
        \includegraphics[width=\textwidth]{img/dataset.pdf}
        % \caption{}
        % \label{fig:my_label}
    \end{figure}
\end{viterbiframe}

\begin{viterbiframe}{Solution}
    \begin{itemize}
        \item Minimize the \alert{worst-case} loss over \alert{groups} in the training data
        \item i.e., Distributionally Robust Optimization \footcite{bental2013robust,oren2019distributionally}
        \item Grouping together \alert{contradictory} sentences with \alert{no negation} words in the \emph{natural language inference} (NLI) example above.
        
    \end{itemize}
\end{viterbiframe}

\begin{viterbiframe}{{Preliminary Experiments, DRO doesn't work}}
\begin{itemize}
    \item DRO in the context of \alert{overparameterized} neural networks in 3 applications
    \begin{itemize}
        \item NLI with the MultiNLI dataset %\footcite{williams2018broad}
        \item Facial recognition with CelebA %\footcite{liu2015deep}
        \item Bird photograph recognition with CUB dataset %\footcite{wah2011cub}.
    \end{itemize}
    \item Problem: if a model achieves \alert{zero training loss}, then it is optimal on both the worst-case (DRO) and the average training objectives:
    \begin{itemize}
        \item For both ERM and group DRO, the \alert{generalization gap} is small on average but large for the worst group.
    \end{itemize}
\end{itemize}
     
\end{viterbiframe}

\begin{viterbiframe}{Solution}

\begin{itemize}
    \item Use \alert{strongly-regularized} group DRO
    \pause
    \item Bi-product: Regularization might not be important for good \alert{average performance} e.g., models can \alert{“train longer and generalize better”} on average \footcite{hoffer2017train} 
    \pause
    \item It appears important for good worst-case performance.
\end{itemize}
\end{viterbiframe}

\section{Problem Setup}

\begin{viterbiframe}{{Objective Function}}
\begin{itemize}
    \item Empirical risk minimization (ERM):
    \begin{align}%\label{eqn:hterm}
      \hterm := \arg\min_{\theta \in \Theta} 
      \ \E_{(x, y) \sim \hP}
    [\ell(\theta; (x, y))]
    \end{align}
    where $\hP$ is the \alert{empirical distribution} over the training data.
    \end{itemize}
\end{viterbiframe}

\begin{viterbiframe}{{Objective Function}}
\begin{itemize}
    \item Distributionally Robust Optimization (DRO):
        \begin{align}\label{eq:dro}
          \underset{\theta \in \Theta}{\vphantom{\sup}\min} \Bigl\{ \mathcal{R}(\theta) := \sup_{Q \in \mathcal{Q}} \E_{(x, y) \sim Q}[\ell(\theta; (x, y))] \Bigr\}.
        \end{align}
        \pause
    The uncertainty set $\mathcal{Q}$ encodes the \alert{possible test distributions} that we want our model to perform well on.
    \end{itemize}
    \begin{figure}
        \centering
        \includegraphics[width=.4\linewidth]{img/mmodal.png}
        % \caption{Caption}
        % \label{fig:my_label}
    \end{figure}
\end{viterbiframe}

\begin{viterbiframe}{Uncertainty Set}

\begin{itemize}
 \item Use \alert{Prior knowledge} of spurious correlations to \alert{define groups} over the training data
 
 \pause
 \item \alert{Group DRO}: the training distribution $P$ is assumed to be a mixture of $m$ groups $P_g$ indexed by $\mathcal{G} = \{1, 2, \ldots, m\}$.
 \pause
 \item Then 
 \[
 \mathcal{Q} := \{ \sum_{g=1}^m q_g P_g: q \in \Delta_m \}
 \]
 \pause
 \item and the worst-case risk is
 \begin{align}
\label{eqn:groupdroq}
  \mathcal{R}(\theta)
  &= \underset{g \in \mathcal{G}}{\max} \ \E_{(x, y) \sim P_g}[\ell(\theta; (x, y))].
\end{align} 
    by concavity of linear function.

\pause
 \item Learns models that are robust to \alert{group shifts}

\end{itemize}
 
\end{viterbiframe}  

\begin{viterbiframe}{Group DRO}
\begin{itemize}
    \item Assume the training data comprises $(x, y, g)$ triplets
    \pause
    \item Though we \alert{don't observe} $g$ at test time, so the model cannot use $g$ directly. 
    \pause
    \item Instead, worst-group risk $\hat{\mathcal{R}}(\theta)$:
    \begin{align}
    % \label{eqn:htdro}
      \htdro := \underset{\theta \in \Theta}{\vphantom{\sup}\argmin} \Bigl\{\hat{\mathcal{R}}(\theta) :=  \underset{g \in \mathcal{G}}{\vphantom{\argmin}\max} \ \E_{(x, y) \sim \hP_g}[\ell(\theta; (x, y))] \Bigr\}\nonumber
    \end{align}
    
    \item Worst-group \emph{generalization gap}
    \[\delta \eqdef  \mathcal{R}(\theta)  -\hat{\mathcal{R}}(\theta)\;,\]
    which for overparameterized neural networks, is large unless we apply \alert{sufficient regularization}.
\end{itemize}
\end{viterbiframe}


\begin{viterbiframe}{Application}
    Each data point $(x, y)$ has some \alert{input attribute} $a(x) \in \mathcal{A}$ that is spuriously correlated with the label $y$
    \pause
    
    We use this prior knowledge to \alert{form $m = |\mathcal{A}| \times |\mathcal{Y}|$} groups, one for each value of $(a, y)$.
    
    \pause
    
    
\end{viterbiframe}

\begin{viterbiframe}{Application}

\begin{itemize}

    \item Object recognition with correlated \alert{backgrounds}: Waterbirds
    
    $\mathcal{Y} = \{\text{waterbird}, \text{landbird}\}$
    
    $\mathcal{A} = \{\text{water background}, \text{land background}\}$,
    
    Waterbirds (landbirds) more frequently appearing against a water (land)
    
    \pause
    \item Object recognition with correlated \alert{demographics}: CelebA dataset
    
    $\mathcal{Y} = \{\text{blond}, \text{dark}\}$
    
    $\mathcal{A} = \{\text{male}, \text{female}\}$
    
    Smallest group: blond-haired males.
\end{itemize}
\end{viterbiframe}

\begin{viterbiframe}{Application (Cont.)}
\begin{itemize}
    \item NLI: MultiNLI dataset:
    
    Given \alert{hypothesis} is entailed by, neutral with, or contradicts a given \alert{premise}
    
    $\mathcal{Y} = \{\text{entailed}, \text{neutral}, \text{contradictory}\}$
    
    $\mathcal{A} = \{\text{no negation}, \text{negation}\}$
    
    Smallest group: entailment with negations
\end{itemize}
\end{viterbiframe}

\section{ERM vs. DRO}

\begin{viterbiframe}{ERM vs. group DRO}
\begin{itemize}
    \item Fine-tune ResNet50 models on Waterbirds and CelebA

\item Fine-tune BERT model on MultiNLI

\item SGD to train ERM

\item Special SGD to train group DRO

\end{itemize}


\end{viterbiframe}

\subsection{Both Have Poor Worst-group Acc}

\begin{viterbiframe}{Both Have Poor Worst-group Acc}

\begin{table}[]
    \centering
    \includegraphics[width=.7\textwidth]{img/table_1.pdf}
    \caption{Cells are colored by accuracy, from low (red) to medium (white) to high (blue) accuracy}
    \label{tab:analysis}
\end{table}


\end{viterbiframe}

\begin{viterbiframe}{Both Have Poor Worst-group Acc}

\begin{figure}
    \centering
    \includegraphics[width=\textwidth]{img/train_curve.pdf}
    \caption{Training (light) and validation (dark) accuracy for CelebA throughout training}
    % \label{fig:my_label}
\end{figure}

% If train using \alert{standard regularization} and hyperparameter settings

\begin{itemize}
    \item ERM and DRO have \alert{poor worst-group} accuracy in the overparameterized regime
    
    \item DRO \alert{improves worst-group} accuracy under appropriate regularization
\end{itemize}
\end{viterbiframe}

\subsection{DRO improves with Appropriate Regularization}
\begin{viterbiframe}{DRO improves with Appropriate Regularization}
    \begin{itemize}
        \item[1)] {\bf $\ell_2$ penalty}: Substantially reduces the \alert{generalization gap} for each group.
        
        Both still achieve high average test accuracies
        
        ERM poor worst-group Acc
        
        DRO improve worst-group Acc
        
        see Table~\ref{tab:analysis}
        
        \pause
        \item[2)] {\bf Early stopping}: Train each model for a \alert{fixed (small) number of epochs}
        
        Same as $\ell_2$
        
        Small \alert{drop} for DRO in Average test Acc
        
        see Table~\ref{tab:analysis}
    \end{itemize}
\end{viterbiframe}

\subsection{Group Adjustments Improves DRO}

\begin{viterbiframe}{Group Adjustments Improves DRO}
    
    \begin{itemize}
    \item
        Even with regularization, the \alert{generalization gap} vary across groups: 
        
        e.g., \alert{Train-test accuracy gap}, Waterbirds, DRO + strong $\Ltwo$ penalty: 
        
        Smallest group: $15.4\%$ 
        
        Largest group: $1.0\%$
        
        \pause
        \item So, at training time, \alert{prioritize} the groups that we \alert{expect to have a larger generalization gap}
        
        $\rightarrow$  Structural Risk Minimization \footcite{vapnik1992principles}.
    
    \end{itemize}
\end{viterbiframe}

\begin{viterbiframe}{Group-adjusted DRO Estimator}
    
    Generalization Gap \[\gengap = \E_{(x,y) \sim P_g}[\ell(\theta; (x,y))] - \E_{(x,y) \sim \hP_g}[\ell(\theta; (x,y))].\]
    
    use $\hatgengap=\frac{C}{\mathcal{Q}rt{n_g}}$, then \emph{Group-adjusted} DRO Estimator is
    \begin{align}\label{eqn:htadj}
        \htadj \eqdef \argmin_{\theta \in \Theta} \ \underset{g \in \mathcal{G}}{\vphantom{\argmin}\max} \ \left\{ \E_{(x, y) \sim \hP_g}[\ell(\theta; (x, y))] + \frac{C}{\mathcal{Q}rt{n_g}}\right\}.
    \end{align}
\end{viterbiframe}

\begin{viterbiframe}{Group-adjusting Improves DRO}
    
    \begin{table}
    \includegraphics[width=0.8\textwidth]{img/table_2.pdf}
    % \caption{Average and worst-group test accuracies with and without group adjustments. Group adjustments improve worst-group accuracy, though average accuracy drops for Waterbirds.}
    % \label{tab:adj}
    \end{table}

    \begin{figure}
      \includegraphics[width=\textwidth]{img/group_adj_waterbirds.pdf}
      \caption{
        Training (light) and validation (dark) accuracies for each group over time,
        for different adjustments $C$.
        % When $C=0$, the generalization gap for waterbirds on land (green line) is large, dragging down worst-group accuracy.
        At $C=2$, which has the best worst-group validation accuracy, the accuracies are balanced.
        % At $C=4$, we overcompensate for group sizes, so smaller groups (e.g., waterbirds on land) do better at the expense of larger groups (e.g., landbirds on land).
    }
    \label{fig:adj}
    \end{figure}

\end{viterbiframe}

\section{DRO vs. IW}

\begin{viterbiframe}{DRO vs. Importance Weighting}
    Importance-weighted estimator learn
    \begin{align}
      \htw \eqdef \argmin_{\theta \in \Theta} \ \E_{(x,y,g) \sim \hP}[w_g \ \ell(\theta; (x, y))].
    \end{align}
    for $w \in \Delta_m$.

    \pause
    Take $w_g = 1 / \E_{g^\prime \sim \hP} [\1(g^\prime = g)]$.
\end{viterbiframe}
    
\subsection{Empirical Comparison}
\begin{viterbiframe}{Empirical Comparison}
        \begin{table}[th]
        \includegraphics[width=\textwidth]{img/table_3.pdf}
        \caption{Comparison of ERM, upweighting (UW), and group DRO models.
        % , with binomial standard deviation in parenthesis. For each objective, we grid search over $\Ltwo$ penalty strength, number of epochs, and group adjustments and report on the model with highest validation accuracy.
        % These numbers differ from the previous tables because of the larger grid search.
          }
        \label{tab:benchmark}
        \end{table}
\end{viterbiframe}

\subsection{Theoretical comparison}
\begin{viterbiframe}{Theoretical comparison}

    \begin{itemize}
        \item IW and DRO can learn equivalent models in the \alert{convex} setting under some importance weights 
        
        \pause
        % \item 
        \begin{proposition}\label{prop:convex-reweight}
          Suppose that the loss $\ell(\cdot; z)$ is \alert{continuous} and \alert{convex} for all $z$ in $\mathcal{Z}$,
          and let the uncertainty set $\mathcal{Q}$ be a set of distributions supported on $\mathcal{Z}$.
          Assume that $\mathcal{Q}$ and the model family $\Theta \subseteq \R^d$ are \alert{convex} and \alert{compact},
          and let $\theta^* \in \Theta$ be a \alert{minimizer of the worst-group} objective $\mathcal{R}(\theta)$.
          Then there exists a distribution $Q^* \in \mathcal{Q}$ such that $\theta^* \in \argmin_\theta \E_{z \sim Q^*} [\ell(\theta;z)]$.
        \end{proposition}
        \begin{proof}
        Linear program duality.
        \end{proof}
        
        \item Not necessarily when the models are \alert{non-convex}.
    \end{itemize}
\end{viterbiframe}

\begin{viterbiframe}{Non-convex Case}
Counter Example
\begin{figure}[th]
      \centering
      \includegraphics[width=0.8\textwidth]{img/counterexample.pdf}
      \caption{
        Toy example illustrating that \alert{DRO} and \alert{importance weighting} are \alert{not equivalent}. The DRO solution is $\theta^*$, while any importance weighting would result in solutions at $\theta_1$ or $\theta_2$.
      }\label{fig:DRO_counterexample}
    \end{figure}
\end{viterbiframe}

\section{Algorithm}
\begin{viterbiframe}{Training DRO}
    Prvious work: two \alert{stochastic} optimization methods for \alert{general non-convex} DRO:
    \begin{itemize}
        \item[(i)] Stochastic gradient descent (SGD) on the \alert{Lagrangian dual} of the objective: fails for group DRO 
        \item[(ii)] Direct \alert{minimax} optimization 
    \end{itemize}
    
    Rewrite the DRO as
    \begin{align}
      \min_{\theta \in \Theta} \sup_{q \in \Delta_m} \ \sum_{g=1}^m q_g \E_{(x, y) \sim P_g}[\ell(\theta; (x, y))].
    \end{align}
\end{viterbiframe}

\begin{viterbiframe}{Algorithm}
    \small{
    \begin{algorithm}[H]
    \label{alg:opt}
     \DontPrintSemicolon
     \KwIn{Step sizes $\eta_q, \eta_\theta; P_g$ for each $g\in \gset$}
     Initialize $\theta\iter{0}$ and $\padv\iter{0}$\;
     \For{$t=1,\dots,T$}{
      $ g \sim \text{Uniform}(1,\dots,\ngroup)$ \tcp*{Choose a group $g$ at random}
      $ x,y \sim P_g$ \tcp*{Sample $x, y$ from group $g$}
      $ \padv^\prime \gets \padv\iter{t-1}$;\
      $ \padv^\prime_g \gets \padv^\prime_g\exp({\eta_q\ell(\theta\iter{t-1}; (x,y))})$ \tcp*{Update weights for group $g$}
      $ \padv\iter{t} \gets \padv^\prime / \sum_{g^\prime} \padv^\prime_{g^\prime}$ \tcp*{Renormalize $q$}
      $\theta\iter{t} \gets \theta\iter{t-1} - \eta_\theta \padv_g^{(t)}\nabla \ell(\theta\iter{t-1}; (x,y))$ \tcp*{Use $q$ to update $\theta$}
     }
      \caption{Online optimization algorithm for group DRO}
    \end{algorithm}
    }
    Has a convergence rate $1/\sqrt{T}$ in the convex setting.
\end{viterbiframe}


\section{Conclusion}
\begin{viterbiframe}{Discussion}
    \begin{itemize}
        \item What to do when groups are imperfectly specified? e.g., No prior knowledge, or training groups are ill defined for the test time.
        
        \item Regarding Average vs. Worst-case generalization in NNs, what other tools can you think of to analyze/address the trade-off?
    \end{itemize}
\end{viterbiframe}

\section{Fair w/o Demogs}
\begin{viterbiframe}{{Paper Intro}}
\begin{itemize}
    \item {\bf Fairness Without Demographics in Repeated Loss Minimization}
    \begin{itemize}
        {\small
        \item[] Tatsunori B. Hashimoto, \textit{\scriptsize Stanford University}
        \item[] Megha Srivastava, \textit{\scriptsize Stanford University}
        \item[] Hongseok Namkoong, \textit{\scriptsize Stanford University}
        \item[] Percy Liang, \textit{\scriptsize Stanford University}}
    \end{itemize}
\end{itemize}
% We follow the current of the paper.
\end{viterbiframe}

\begin{viterbiframe}{Introduction}
    \begin{itemize}
        \item Minimize average loss $~\rightarrow~$ Representation disparity
        \pause
        \item Acc effects \emph{use retention}: a minority group can shrink over time.
        \pause
        \item ERM amplifies representation disparity
        \pause
        \item Minimizes the \emph{worst case risk} 
        \pause
        \begin{itemize}
            \item[] over all distributions close to the empirical distribution.
        \end{itemize} 
        \pause
        \item This controls the risk of the minority group
        \pause
        \begin{itemize}
            \item[] while remaining oblivious to the identity of the groups
        \end{itemize} 
        
    \end{itemize}
    
\end{viterbiframe}

\end{document}
